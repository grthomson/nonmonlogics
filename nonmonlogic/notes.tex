\documentclass[a4paper,11pt]{report}
\usepackage{amsmath,amsfonts,amssymb,amsthm,hyperref, marvosym, accents}
%\usepackage{fullpage}
\usepackage{bm}
\usepackage{cleveref}
\usepackage{mdframed}
\usepackage{csquotes}
\usepackage[super]{natbib}
\usepackage{minitoc}
\usepackage{wasysym}
\usepackage{setspace}
\usepackage{enumerate}
\usepackage{enumitem}
\usepackage{tikz-cd}
\usepackage{tikz}

\begin{document}
This note outlines a proposal for a novel mathematical treatment of a phenomenon which is well-recognised in formal semantics (understood as a subfield of linguistics) but which has increasing relevance for denotational semantics (understood as a tool for the analysis and design of programming languages). The discussion here is at a high level and the formal aspects are not developed in any detail, but I refer throughout to relevant mathematical literature and draw heavily on Avron (date), Zeilberger (date), (SUBSTRUCTURAL GUYS), and...

This phenomenon can be roughly described by two claims: i) everyday reasoning as expressed in a natural language such as English is \emph{defeasible}, in the sense that conclusions derived from a set of true premises by a sequence of acceptable rules of inference should sometimes be withheld as new evidence is uncovered and as our premise-set grows and ii) there is a special fragment of reasoning, which we call \emph{deductive} reasoning, characterised by the fact that 

THE RELEVANT MATHEMATICAL IDEAS ARE : SUBSTRUCTURAL LOGICS, CATEGORICAL SEMANTICS, 2-CATEGORY THEORY
\end{document}